\title{Proyecto de Redes - Callcenter - PUJ, Cali.}
\author{Alejandro~Cardona,
        Luis~Santiago~Osorio}

\date{\today}

\documentclass[12pt]{article}
\hoffset - 1cm % 2cm
\voffset - 0.54cm % 2cm
\setlength{\textwidth}{15cm}
\baselinestretch
\renewcommand{\baselinestretch}{1.5}

\usepackage{dcolumn}
\usepackage{colortbl}
\usepackage{graphicx}
\usepackage{rotating}

\begin{document}
\maketitle
\pagebreak

\tableofcontents

\pagebreak 
\section{\textbf{Introduci\'on}}
En el desarrollo de una red para un callcenter, se aplicaran los pasos basicos de diseño, para obtener una solucion que concuerde con los requerimientos nesesarios de dicha red. En el proceso se realizaran los diseños y los calculos de el consumo, espacio y demas, para ayar los elementos fisicos nesesarios y adecuados para el montaje de la red en donde no se haga desperdicio de recurso y se empleen los equipos y los elementos adecuados.\\\\

En este documento se mostrara el plano de la red, los equipos nesesarios, tablas de comparacion y demas, que respalden el buen diseño de la red y la buena seleccion de los equipos.

\pagebreak
\section{\textbf{Planteamiento del problema}}
Se nesecita realizar el diseño de una red para un callcenter, equipado por 18 computadores que seran manejados por personal de la empresa, 2 computadores manejados en la oficina de direccion, servidores, y un servicio wifi, para la sala de visitas. En donde todos los computadores y dispositivos moviles, deberan poder tener acceso a internet, con sus debidas restricciones, en cuanto a el acceso a las diferentes redes internas del callcenter.\\\\
Ademas se debe garantizar una robustes en la red, ya que la conexion debe ser permanente para que no sea afectado el trabajo del callcenter.\\\\
Dentro de los problemas basicos que se abordaran seran los siguentes:
\begin{itemize}
\item
Probedor de servicios
\item
Computadores
\item
infraestructura
\item
Robustes
\end{itemize}

\pagebreak
\section{\textbf{Requerimienos de la red}}
Los requerimientos de la red para el callcenter seran los siguentes:
\begin{itemize}
\item
Numero de computadores.
\item
Conectividad Voip.
\item
Usuarios.
\item
Provedor de internet.
\item
Tipos de cables.
\item
Velocidad.
\item
Costos.
\end{itemize}

\subsection{\textbf{Numero de computadores}}
En la red se tendran al rededor de 30 a 42, con la siguente distribucion:
\begin{itemize}
\item
2-7 computadores en la oficina, 2 con punto de red y de 1 a 5 por acceso wifi.
\item
10 posibes accesos en la salada de espera, todos por acceso wifi.
\item
18 computadores en la sala de atencion del callcenter, cada uno con su respectivo punto de red.
\item
5 puntos de cceso a camaras de seguridad.
\end{itemize}

\subsection{\textbf{Conectividad VoIP}}
Debe existir conectividad Voip, tanto en los computadores de servicio como en los computadores de la oficina, ya que debido a que el software que se manejara para la atencion del callcenter maneja la conectividad por VoIP.

%Analizar para terminar la tabla
\subsection{\textbf{Usuarios}}
Los usuarios de la red del callcenter, tendran la capacidad de desarrollar diferentes tareas y diferentes tipos de servicios en la red, los cuales se muestran acontinacion.\\\\
\begin{tabular}{|c|c|c|c|c|}
\hline
\makebox[3.1cm][c]{\textbf{Servicio/Usuario}} &\makebox[2.7cm][c]{\textbf{Pcs-Atencion}} &\makebox[2.7cm][c]{\textbf{Pcs-Oficina}} &\makebox[2.7cm][c]{\textbf{Pcs-Sala}} &\makebox[2.7cm][c]{\textbf{Seguridad}}\\
\hline
\makebox[2.7cm][c]{Nevegacion Web} &\makebox[2.7cm][c]{No} &\makebox[2.7cm][c]{Si} &\makebox[2.7cm][c]{Si} &\makebox[2.7cm][c]{No}\\
\hline
\makebox[2.7cm][c]{VoIp} &\makebox[2.7cm][c]{Si} &\makebox[2.7cm][c]{Si} &\makebox[2.7cm][c]{No} &\makebox[2.7cm][c]{No}\\
\hline
\makebox[2.7cm][c]{Descargas} &\makebox[2.7cm][c]{No} &\makebox[2.7cm][c]{Si} &\makebox[2.7cm][c]{Si} &\makebox[2.7cm][c]{No}\\
\hline
\makebox[2.7cm][c]{Video llamadas} &\makebox[2.7cm][c]{Si} &\makebox[2.7cm][c]{Si} &\makebox[2.7cm][c]{No} &\makebox[2.7cm][c]{No}\\
\hline
\makebox[2.7cm][c]{OS} &\makebox[2.7cm][c]{Windows} &\makebox[2.7cm][c]{Windows} &\makebox[2.7cm][c]{All} &\makebox[2.7cm][c]{Otro}\\
\hline
\end{tabular}

\subsection{\textbf{Probedor de internet}}
Se requiere que se tengan 2 probedores de internet para que exista una robustes en la red, ya que debido a que es un callcenter, la coneccion debera ser continua y sin interrupciones, tambien se debe tener en cuenta la calidad del servicio del probedor en cuanto a una respuesta afallos en la red y si el servicio cumple con las nesecidades de los usuarios.

\subsection{\textbf{Cables}}
Los tipos de cables para la red, deben soportar el trafico de red y garantizar coneccion.

\subsection{\textbf{Velocidad}}
La veloidad de coneccion debe ser efectiva y continua, para poder sostener las llamadaas del callcenter, y que siempre exista una comunicacion fluida con los clientes del callcenter.

\subsection{\textbf{Costo}}
El costo de la red es libre, se cuenta con el capital para cualquier invercion, siempre y cuando esta inversion este justificada.

\pagebreak
\section{\textbf{Analizis de consumo}}
Dentro de los requerimientos de la red, se tendran en cuenta para el analizis del trafico aquellos en los que aya un mayor uso por parte de los usuarios del callcenter, para poder posteriormente, hacer las posibles evaluaciones de seleccion del probedor de servicio, seleccion de los equipos a utilizar y de los diferentes materiales de infraestructura que se utilizaran.\\\\ 
En los servicios que se deben prestar para el callcenter los mas utilizados son el manejo de VoIP y las camaras de seguridad debido a que tienen un flujo continuo sobre la red, y la navegacion web para los computadores que no estan restringidos.\\\\
%% Para el analizis del consumo se tendran en cuenta los diferentes tipos de servicio que se le ofreceran a los usuarios de la red.\\\\
%% Los servicios son los siguentes:
%% \begin{itemize}
%% \item
%% Navegacion Web.
%% \item
%% VoIP.
%% \item
%% Descargas.
%% \item
%% Video llamadas.
%% \item
%% Camaras de seguridad.
%% \end{itemize}

\subsection{\textbf{Consumo VoIP}}
En el consumo VoIP, se tiene en cuenta que este servicio consta de 2 etapas, la señalisacion de la llamada y la transmision de audio que es realizada a traves de RTP, dado a que el ancho de banda consumido por la señalizacion no es relevante, se enfocara el calculo de consumo en la transmicion del audio.\\\\
Para el analizis de esta transmicion veremos el empaquetamiento de los datos en las 7 capas del modelo OSI. El audio codificado necesita ser empaquetado dentro de paquetes RTP. A su vez, los paquetes RTP necesitan ser empaquetados dentro de paquetes UDP, que luego necesitan ser empaquetados dentro de paquetes IP. en este ejemplo tomaremos Ethernet que es el tipo de red más común, y requiere otro empaquetamiento.\\\\
En la siguente tabla se ilustra lo dicho, con los respectivos valores para cada una de las capaz.\\\\

\begin{tabular}{|c|c|}
\hline
\makebox[3.1cm][c]{Ethernet} &\makebox[3.1cm][c]{15.2 kbps}\\
\hline
\makebox[2.7cm][c]{IP} &\makebox[3.1cm][c]{8 kbps}\\
\hline
\makebox[2.7cm][c]{UDP} &\makebox[3.1cm][c]{3.2 kbps}\\
\hline
\makebox[2.7cm][c]{RTP} &\makebox[3.1cm][c]{4.8 kbps}\\
\hline
\makebox[2.7cm][c]{Encoded Audio} &\makebox[3.1cm][c]{Depende del codec}\\
\hline
\end{tabular}\\\\\\
Los codecs de audio para el VoIP, son el G711, G722, GSM Y G729 en los cuales veremos diferentes caractertizticas y tomaremos el mas indicado para el callcenter.\\\\

\begin{tabular}{|c|c|c|c|}
\hline
\makebox[3.1cm][c]{\textbf{Codec}} &\makebox[3.1cm][c]{\textbf{Calidad Audio}} &\makebox[3.1cm][c]{\textbf{Recursos CPU}} &\makebox[3.1cm][c]{\textbf{Tamaño}}\\
\hline
\makebox[2.7cm][c]{G711} &\makebox[3.1cm][c]{Buena} &\makebox[3.1cm][c]{Muy pocos} &\makebox[3.1cm][c]{95.2}\\
\hline
\makebox[2.7cm][c]{G722} &\makebox[3.1cm][c]{Muy Buena} &\makebox[3.1cm][c]{Pocos} &\makebox[3.1cm][c]{95.2}\\
\hline
\makebox[2.7cm][c]{GSM} &\makebox[3.1cm][c]{Aceptable} &\makebox[3.1cm][c]{Promedio} &\makebox[3.1cm][c]{44.2}\\
\hline
\makebox[2.7cm][c]{G729} &\makebox[3.1cm][c]{Promedio} &\makebox[3.1cm][c]{Altos} &\makebox[3.1cm][c]{39.2}\\
\hline
\end{tabular}\\\\
 
Segun las tablas anteriormente mostradas, el codec adecuado para el callcenter sera el G722, ya que uno de los principales requerimientos es que debe haber una buena comunicacion, se utilizara este codec y debido a que se cuenta con los recursos nesesarios se adaptara la red pra tener el uso de este codec.\\\\
En un total el consumo de uno de los equipos para el uso de este servicio seria el siguente en cuanto a kbps:\\\\

\begin{tabular}{|c|c|}
\hline
\makebox[3.1cm][c]{Ethernet} &\makebox[3.1cm][c]{15.2 kbps}\\
\hline
\makebox[2.7cm][c]{IP} &\makebox[3.1cm][c]{8 kbps}\\
\hline
\makebox[2.7cm][c]{UDP} &\makebox[3.1cm][c]{3.2 kbps}\\
\hline
\makebox[2.7cm][c]{RTP} &\makebox[3.1cm][c]{4.8 kbps}\\
\hline
\makebox[2.7cm][c]{Encoded Audio} &\makebox[3.1cm][c]{64 kbps}\\
\hline
\makebox[2.7cm][c]{\textbf{Total}} &\makebox[3.1cm][c]{\textbf{95.2 kbps}}\\
\hline
\end{tabular}\\\\\\
De acuerdo con los calculos, cada equipo de los usuarios de atencion y de los usuarios de oficina consumira un total de 73 kbps, y teniendo encuenta la cacidad media de coneccion de equipos en la red el consumo por kbps seria de 25 * 73 kbps dandonos un total medio de consumo de VoIP de 2380 kbps.

\subsection{\textbf{Consumo TCP/IP Camaras de seguridad}}
%% Los sistemas de vigilancia IP utilizan cada vez más recursos de las redes TCP/IP para enviar video de un punto a otro (por ejemplo, de una Cámara IP a un sistema de Gestión de Video típicamente en una computadora personal o un servidor dedicado) Las redes de IP son un medio atractivo para transportar video porque por un cable pueden llevar el video de muchas cámaras. Estas redes también sirven un propósito multi-funcional porque el mismo cable puede llevar video así como audio, señales de alarma, señales de relevo, órdenes de PTZ, y los datos de serie. Con POE (Power Over Ethernet), el cable puede llevar también la electricidad a la cámara. Esto simplifica mucho la cantidad de cables que se requirió anteriormente. Sin embargo, la red TCP/IP es limitada por la cantidad de tráfico que puede llevar – conocido como ancho de banda, que es medida (bps). La siguiente tabla muestra las tres clases de redes de Ethernet de estrella-topología de uso corriente hoy.

El sistema de vigilancia del callcenter sera de tipo IP, el cual utilizara los recursos TCP/IP de la red para enviar video y audio de cada camara a el servidor dedicado del sistema de vigilancia. el calculo de ancho de banda para las camaraws de seguridad esta dado, segun la resolucioon que envien las camaras al servidor. El sistema de seguridad del callcenter se realizara en formato de video MPEG4. En la siguente tabla se muestra algunas de las resoluciones y sus valores de consumo de banda ancha para el formato MPEG4.\\\\
\begin{tabular}{|c|c|c|}
\hline
\makebox[3.1cm][c]{\textbf{Resolucion}} &\makebox[3.1cm][c]{\textbf{IPS}} &\makebox[3.1cm][c]{\textbf{Kbps}}\\
\hline
\makebox[3.1cm][c]{CIF} &\makebox[3.1cm][c]{3} &\makebox[3.1cm][c]{160}\\
\hline
\makebox[3.1cm][c]{CIF} &\makebox[3.1cm][c]{7} &\makebox[3.1cm][c]{185}\\
\hline
\makebox[3.1cm][c]{CIF} &\makebox[3.1cm][c]{15} &\makebox[3.1cm][c]{200}\\
\hline
\makebox[3.1cm][c]{CIF} &\makebox[3.1cm][c]{30} &\makebox[3.1cm][c]{500}\\
\hline
\makebox[3.1cm][c]{2CIF} &\makebox[3.1cm][c]{3} &\makebox[3.1cm][c]{320}\\
\hline
\makebox[3.1cm][c]{2CIF} &\makebox[3.1cm][c]{7} &\makebox[3.1cm][c]{370}\\
\hline
\makebox[3.1cm][c]{2CIF} &\makebox[3.1cm][c]{15} &\makebox[3.1cm][c]{400}\\
\hline
\makebox[3.1cm][c]{2CIF} &\makebox[3.1cm][c]{30} &\makebox[3.1cm][c]{1000}\\
\hline
\makebox[3.1cm][c]{4CIF} &\makebox[3.1cm][c]{3} &\makebox[3.1cm][c]{640}\\
\hline
\makebox[3.1cm][c]{4CIF} &\makebox[3.1cm][c]{7} &\makebox[3.1cm][c]{740}\\
\hline
\makebox[3.1cm][c]{4CIF} &\makebox[3.1cm][c]{15} &\makebox[3.1cm][c]{800}\\
\hline
\makebox[3.1cm][c]{4CIF} &\makebox[3.1cm][c]{30} &\makebox[3.1cm][c]{2000}\\
\hline
\end{tabular}\\\\\\

La calidad requerida para el callcenter sera 2CIF a 15 IPS, que es el formato mas comun y de buena calidad para la imagen. Dados los datos el consumo de las camaras de seguridad en la red sera de 400 kbps por cada camara, dandonos un tota de 5 * 400 la suma de el trafico de todas las camaras de seguridad, para un total de 2000 kbps.

\subsection{\textbf{Consumo Navegacion web, chat, videos, email}}
El calculo de el trafico consumido por la navegacion web, chat, videos y email se realizara con un simulador Capsa de Colasoft, en el cual se realizo una medicion de un solo ordenador efectuando lsa tareas descritas. El resultado fue que el computador realizando estas tareas tiene un conumo de alrededor de 36 kbps, lo cual nos dara un total de consumo para los computadores que no tienen esta restriccion de 17 * 36 kbps tomando el peor de los casos en los que esten todos los pcs conectados sin restriccion para un total de 612 kbps.

\subsection{\textbf{Consumo Total}}
El consumo total estara dado por la sumo del consumo de todos los servicios calculados anteriormente. En la siguente tabla de consumo se muestra el total de consumo promedio de la red.\\\\
\begin{tabular}{|c|c|c|}
\hline
\makebox[3.1cm][c]{\textbf{Servicio}} &\makebox[3.1cm][c]{\textbf{1PC}} &\makebox[3.1cm][c]{\textbf{Total}}\\
\hline
\makebox[3.1cm][c]{VoIP} &\makebox[3.1cm][c]{95.2 kbps} &\makebox[3.1cm][c]{2380 kbps}\\
\hline
\makebox[3.1cm][c]{TCP/IP Seguridad} &\makebox[3.1cm][c]{400 kbps} &\makebox[3.1cm][c]{2000 kbps}\\
\hline
\makebox[3.1cm][c]{Navegacion y otros} &\makebox[3.1cm][c]{36 kbps} &\makebox[3.1cm][c]{612 kbps}\\
\hline
\makebox[3.1cm][c]{} &\makebox[3.1cm][c]{\textbf{Total}} &\makebox[3.1cm][c]{4992 kbps}\\
\hline
\end{tabular}\\\\\\

\bibliographystyle{abbrv}
\bibliography{main}

\end{document}

