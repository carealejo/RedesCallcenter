\title{Proyecto de Redes - Callcenter - PUJ, Cali.}
\author{Alejandro~Cardona,
        Luis~Santiago~Osorio}

\date{\today}

\documentclass[12pt]{article}
\hoffset - 1cm % 2cm
\voffset - 0.54cm % 2cm
\setlength{\textwidth}{15cm}
\baselinestretch
\renewcommand{\baselinestretch}{2}

\usepackage{dcolumn}
\usepackage{colortbl}
\usepackage{graphicx}
\usepackage{rotating}

\begin{document}
\maketitle
\pagebreak

\tableofcontents

\pagebreak 
\section{\textbf{Introduci\'on}}
En el desarrollo de una red para un callcenter, se aplicaran los pasos basicos de diseño, para obtener una solucion que concuerde con los requerimientos nesesarios de dicha red. En el proceso se realizaran los diseños y los calculos de el consumo, espacio y demas, para ayar los elementos fisicos nesesarios y adecuados para el montaje de la red en donde no se haga desperdicio de recurso y se empleen los equipos y los elementos adecuados.\\\\

En este documento se mostrara el plano de la red, los equipos nesesarios, tablas de comparacion y demas, que respalden el buen diseño de la red y la buena seleccion de los equipos.

\pagebreak
\section{\textbf{Planteamiento del problema}}
Se nesecita realizar el diseño de una red para un callcenter, equipado por 18 computadores que seran manejados por personal de la empresa, 2 computadores manejados en la oficina de direccion, servidores, y un servicio wifi, para la sala de visitas. En donde todos los computadores y dispositivos moviles, deberan poder tener acceso a internet, con sus debidas restricciones, en cuanto a el acceso a las diferentes redes internas del callcenter.\\\\
Ademas se debe garantizar una robustes en la red, ya que la conexion debe ser permanente para que no sea afectado el trabajo del callcenter.\\\\
Dentro de los problemas basicos que se abordaran seran los siguentes:
\begin{itemize}
\item
Probedor de servicios
\item
Computadores
\item
infraestructura
\item
Robustes
\end{itemize}

\pagebreak
\section{\textbf{Requerimienos de la red}}
Los requerimientos de la red para el callcenter seran los siguentes:
\begin{itemize}
\item
Numero de computadores.
\item
Usuarios.
\item
Provedor de internet.
\item
Tipos de cables.
\item
Velocidad.
\item
Costos.
\end{itemize}

\subsection{\textbf{Numero de computadores}}
En la red se tendran al rededor de 30 a 42, con la siguente distribucion:
\begin{itemize}
\item
2-7 computadores en la oficina, 2 con punto de red y de 1 a 5 por acceso wifi.
\item
10 posibes accesos en la salada de espera, todos por acceso wifi.
\item
18 computadores en la sala de atencion del callcenter, cada uno con su respectivo punto de red.
\item
5 puntos de cceso a camaras de seguridad.
\end{itemize}

%Analizar para terminar la tabla
\subsection{\textbf{Usuarios}}
Los usuarios de la red del callcenter, tendran la capacidad de desarrollar diferentes tareas y diferentes tipos de servicios en la red, los cuales se muestran acontinacion.\\\\
\begin{tabular}{|c|c|c|c|c|}
\hline
\makebox[3.1cm][c]{\textbf{Servicio/Usuario}} &\makebox[2.7cm][c]{\textbf{Pcs-Atencion}} &\makebox[2.7cm][c]{\textbf{Pcs-Oficina}} &\makebox[2.7cm][c]{\textbf{Pcs-Sala}} &\makebox[2.7cm][c]{\textbf{Seguridad}}\\
\hline
\makebox[2.7cm][c]{Nevegacion Web} &\makebox[2.7cm][c]{No} &\makebox[2.7cm][c]{Si} &\makebox[2.7cm][c]{Si} &\makebox[2.7cm][c]{No}\\
\hline
\makebox[2.7cm][c]{VoIp} &\makebox[2.7cm][c]{Si} &\makebox[2.7cm][c]{Si} &\makebox[2.7cm][c]{No} &\makebox[2.7cm][c]{No}\\
\hline
\makebox[2.7cm][c]{Descargas} &\makebox[2.7cm][c]{No} &\makebox[2.7cm][c]{Si} &\makebox[2.7cm][c]{Si} &\makebox[2.7cm][c]{No}\\
\hline
\makebox[2.7cm][c]{Video llamadas} &\makebox[2.7cm][c]{Si} &\makebox[2.7cm][c]{Si} &\makebox[2.7cm][c]{No} &\makebox[2.7cm][c]{No}\\
\hline
\makebox[2.7cm][c]{OS} &\makebox[2.7cm][c]{Windows} &\makebox[2.7cm][c]{Windows} &\makebox[2.7cm][c]{All} &\makebox[2.7cm][c]{Otro}\\
\hline
\end{tabular}

\subsection{\textbf{Probedor de internet}}
Se requiere que se tengan 2 probedores de internet para que exista una robustes en la red, ya que debido a que es un callcenter, la coneccion debera ser continua y sin interrupciones, tambien se debe tener en cuenta la calidad del servicio del probedor en cuanto a una respuesta afallos en la red y si el servicio cumple con las nesecidades de los usuarios.

\subsection{\textbf{Cables}}
Los tipos de cables para la red, deben soportar el trafico de red y garantizar coneccion.

\subsection{\textbf{Velocidad}}
La veloidad de coneccion debe ser efectiva y continua, para poder sostener las llamadaas del callcenter, y que siempre exista una comunicacion fluida con los clientes del callcenter.

\subsection{\textbf{Costo}}
El costo de la red es libre, se cuenta con el capital para cualquier invercion, siempre y cuando esta inversion este justificada.

\pagebreak
\section{\textbf{Analizis de consumo}}
Para el analizis del consumo se tendran en cuenta los diferentes tipos de servicio que se le ofreceran a los usuarios de la red y el tiempo promedio de consumo de cada servicio.\\\\
Los servicios son los siguentes:
\begin{itemize}
\item
Navegacion Web.
\item
VoIP.
\item
Descargas.
\item
Video llamadas.
\item
Camaras de seguridad.
\end{itemize}

\subsection{\textbf{Navegacion Web}}

\bibliographystyle{abbrv}
\bibliography{main}

\end{document}

